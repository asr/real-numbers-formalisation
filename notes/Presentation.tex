\documentclass{beamer}
\usepackage{ragged2e}
\title{Formalization of the Real Numbers}
\author{{{ \rm { \rm (Work in Progress)} \\ {Jorge O. Acevedo-Acosta and Jose Luis Echeverri-Jurado}}}  \\
\;\;\;\;\; {{ \rm Eafit University}}}
\date{{{ \rm Logic and Computation Seminar}} \\ {{ \rm EAFIT University}} \\ {{ \rm 28 de Julio 2015}}}
\usepackage{xcolor}
\xdefinecolor{lavanda}{rgb}{0.8,0.6,1}
\xdefinecolor{oliva}{cmyk}{0.64,0,0.95,0.4}
\xdefinecolor{minaranja}{rgb}{0.94,0.48,0.2}
\usetheme{Warsaw}
\usecolortheme[named=oliva]{structure}
\usepackage{amssymb}
%\usepackage{math}
%\usetheme{Warsaw}
%Preambulo
\begin{document}
\justifying
\begin{frame}
\titlepage
\end{frame}

\begin{frame}
\frametitle{Real Numbers} \framesubtitle{Introduction}
 { \rm The
  formalization of real numbers can be performed forms different, but
  we will continue this work in the axiomatic. First talk about Agda
  as a main tool to develop this work and secondly the axiomatization
  of the real numbers, drawing a parallel between writing as is done
  in classical mathematics and the code used in Agda.}
\end{frame}

\begin{frame}
\frametitle{Agda} \framesubtitle{What is agda?}
{ \rm ``A dependently
  typed programming language and proof assistant. Agda is a functional
  programming language with dependent types. It is an extension of
  Martin-L\"of’s intuitionistic type theory.'' \footnote{{{ \tiny \rm Ana
      Bove, Peter Dybjer, and Ulf Norell. A Brief Overview of Agda –
A Functional Language with Dependent Types. pp. 1}}} Agda It has infix,
mixfix operators, Unicode characters, and an interactive Emacs
interface.\\}\;\;\;\;

{ \rm ``Agda is based on intuitionistic type theory.''\footnote{{{ \tiny \rm
      Ulf Norell and James Chapman. Dependently Typed Programming in
      Agda. pp. 1-2}}} A great contribution from the constructive
  mathematics by mathematical Per Martin-L\"of. Agda has many
  similarities with other proof assistants based on dependent types,
  such as Coq and others.}
\end{frame}

\begin{frame}
\frametitle{Agda} \framesubtitle{Developers agda}
{ \rm ``Agda is
  open-source and enjoys contributions from many authors. The center
  of the Agda development is the Programming Logic group at Chalmers
  and Gothenburg University; the main developers are Ulf Norell, Nils
  Anders Danielsson, and Andreas Abel.'' \footnote{{{ \tiny \rm Agda wiki
      page. http://wiki.portal.chalmers.se/agda/pmwiki.php?n=Main.Documentation
}}}}
\end{frame}

\begin{frame}
\frametitle{Agda} \framesubtitle{Datatypes and pattern matching}
{ \rm Datatypes are introduced by a \fcolorbox{orange}{orange}{ \color{white} data} declaration, giving the name and type of the datatype as well as the constructors and their types. \\\

data Bool : Set where\footnote{{{ \tiny \rm
      Ana Bove and Peter Dybjer. Dependent Types at Work. pp. 5}}} \\

~~~true  : Bool \\
~~~false : Bool \\\

The type of Bool is \fcolorbox{orange}{orange}{ \color{white} Set} \\

Functions over Bool can be defined by pattern matching \\\

$not : Bool \to Bool$ \\

not true  = false \\
not false = true} \\
\end{frame}

\begin{frame}
\frametitle{Agda}
\framesubtitle{Datatypes and pattern matching}
{ \rm Another example where we use datatypes is a definition of equality, then use it in the equational reasoning \\\

$data\; \_\equiv\_ :\ \mathbb{R} \to \mathbb{R} \to Set\; where$ \\
$~~~refl :\ \{x : \mathbb{R} \} \to x \equiv x$\footnote{{{ \tiny \rm
      Ana Bove and Peter Dybjer. Dependent Types at Work. pp. 23}}} \\\

$subst\; :\; (P\; :\;\mathbb{R}\; \to\; Set) \to \; \{x\; y\; :\; \mathbb{R}\} \to x \equiv y \to P\; x \to P\; y$\footnote{{{ \tiny \rm
      Ana Bove and Peter Dybjer. Dependent Types at Work. pp. 23}}}\\

$subst\; P\; refl\; Px\; =\; Px$
}
\end{frame}
\begin{frame}
\frametitle{Agda}
\framesubtitle{Reasoning Equational}
$\_ \equiv \langle\_\rangle\_ :\ \forall\ x\; \{y\ z\} \to x \equiv y \to y \equiv z \to x \equiv z$ \\

$\_\; \equiv\langle x\equiv y \rangle\;\; y\equiv z\; =\;\; \equiv-trans\; x\equiv y\;\; y\equiv z$\\\

$\_\centerdot\; :\; \forall\; x \to x\; \equiv\; x$\\
$\_\centerdot\; \_ = refl$
\end{frame}

\begin{frame}
\frametitle{Agda} \framesubtitle{Logic} { \rm In Agda, we can define
  infix and mix-fix operators. One indicates the places of the
  arguments of the operators with underscore ``\_ `` For example,
  disjunction on truth values is usually an infix operator and it can
  be declared in Agda as follows: \\\

$\_ or \_ : Bool \to Bool \to Bool$\\
$true\; or\; \_\; =\; true$ \\
$\_ \;or\; true\; =\; true$ \\
$\_\; or\; \_\;\;\;\;\;\;=\; false$\footnote{{{ \tiny \rm
      Ana Bove and Peter Dybjer. Dependent Types at Work. pp. 6}}} \\\

$\neg\_\; :\; Set\; \to\; Set$\\
$\neg\; A\; =\; A\; \to\; \perp$\footnote{{{ \tiny \rm
      Ana Bove and Peter Dybjer. Dependent Types at Work. pp. 20}}}

}
\end{frame}

\begin{frame}
\frametitle{Agda}
\framesubtitle{infix Inductive}
{ \rm We can define the precedence and association of infix operators.\\\

$infixl$ $60 \; \_ or \_$ \\
$infixr$ $2 \; \_ \wedge \_$}\\\

$data \; \_ \wedge \_ \; (A \; B : Set) : Set \; where$\\
$\;\;$ $ \_,\_ : A \to B \to A \wedge B$

\end{frame}

\begin{frame}
\frametitle{Agda}
\framesubtitle{Natural numbers}
{ \rm Another useful datatype is the type of (unary) natural numbers. \\\

data $Nat$ : Set where \\
$zero : Nat$\\
$succ  : Nat \to Nat$\\\

Addition on natural numbers can be defined as a recursive function. \\\

$+ : Nat \to Nat \to Nat$ \\
$zero  + m = m$ \\
$succ n + m = succ (n + m)$ \\
}
\end{frame}

\begin{frame}
\frametitle{Agda}
\framesubtitle{Natural numbers}
{ \rm Definition of multiplication \\
$* : Nat \to Nat \to Nat$\\
$zero  * m = zero$\\
$succ n * m = (m + n) * m$\\
}
\end{frame}

\begin{frame}
\frametitle{Axiomatic of the Real numbers}
\framesubtitle{Real numbers}

{ \rm "To Hilbertthe demonstration is a chain of affirmations,
built following strict rules ensures that the statement is deduced
from the axioms employees"~\ (Quoted by Oostra, A., 1999).}\\\

{ \rm The axiomatic presentation of Real numbers originally made
Hilbert identifies four groups of axioms: axioms connection,
calculation, ordination and completeness.}\\\

{ \rm In some modern texts Mathematical~\cite{Rosenlicht} Analysis an axiomatic presentation of Real
numbers is done, which consists of a non-defined terms not susceptible
primitive concepts of definition and a set of axioms that we accept as
true, this is what constitutes the starting point a mathematical
theory. Cadential manner of starting to testing a series of
properties, theorems from the axioms and/or other properties
previously demonstrated.
}
\end{frame}

\begin{frame}
\frametitle{Axiomatization of Real numbers}
\framesubtitle{Real numbers}

{ \rm The system of Real numbers as the set $\Re$ is defined with two
basic operations called:\\\
}
Addition\\\
$\_ +\_ : ~\Re ~\to ~\Re ~\to ~\Re$\\\

Multiplication\\\
$\_*\_ : ~\Re ~\to ~\Re ~\to ~\Re$\\\

And order relation called ``greather than'':\\\

$\_>\_ : ~\Re ~\to ~\Re \to ~Set$\\\

{ \rm Besides complying axioms, pose some
definitions and then the properties are checked (theorems).
}
\end{frame}

\begin{frame}
\frametitle{Axiomatization of Real numbers}
\framesubtitle{Real numbers}
{ \rm The axioms are classified into three groups: field axioms, axioms of
order and completeness axiom ~\cite{Rosenlicht}. In the constructivist math
~\cite{Bridges1999} an axiom more is entered. the Archimedean property which
will be discussed later.
\\
Field axioms are those that satisfy the defined operations
plus (+) and multiplication (*).\\\
}
Addition axioms.
\\
\begin{align*}
  +-comm : (x\ y : \Re)    & \to ~ x + y ~\equiv\ y + x\\
  +-asso : (x\ y\ z : \Re) & \to x + y + z ~\equiv\ x + (y + z)\\
  +-neut : (x : \Re)       & \to    x + r_{0} ~\equiv\ x\\
  +-inve : (x : \Re)       & \to x + (- x) ~\equiv\ r_{0}
\end{align*}
\\
Where:
\\
$-\_  : \Re ~\to ~\Re$
\\
$\_ ^{-1} : \Re ~\to ~\Re$

\end{frame}

\begin{frame}
\frametitle{Axiomatization of Real numbers}
\framesubtitle{Real numbers}

Multiplication axioms.\\\

\begin{align*}
  *-comm : (x\ y : \Re)    & ~\to\ x * y\          \equiv\ y * x\\
  *-asso : (x\ y\ z : \Re) & ~\to\ x * y * z\      \equiv\ x * (y * z)\\
  *-neut : (x : \Re)       & ~\to\ x * r_{1}\       \equiv x\\
  *-inve : (x : \Re)       & ~\to\ ~\lnot\ (x~\equiv\ r_{0})~\to\ x * (x^{-1})\ \equiv\ r_{1}\\\
\end{align*}

Distributivity axiom.\\\

  *-dist : $(x\ y\ z : \Re) ~\to\ x * (y + z) ~\equiv\ x * y + x * z$

\end{frame}

\begin{frame}
\frametitle{Axiomatization of Real numbers}
\framesubtitle{Real numbers}
{ \rm The axioms of order are those that satisfy the relationship of order $(>)$\\\
}
Order axioms.\\\
\begin{align*}
  >-asym     : \lbrace x\ y    : \Re \rbrace &\to\ x > y ~\to\ \lnot\ (y > x)\\
  >-trans    : \lbrace x\ y\ z : \Re \rbrace &\to\ x > y ~\to\ y > z ~\to\ x > z\\
  >-+-cong-l : \lbrace x\ y\ z : \Re \rbrace & \to\ x > y ~\to\ z + x > z + y\\
  >-*-cong-l : \lbrace x\ y\ z : \Re \rbrace & \to\ z > r_0 ~\to\ x > y ~\to\ z * x > z * y\\
  trichotomy : (x\ y : \Re) & \to\ (x > y) ~\lor\ (x ~\equiv\ y) ~\lor\ (x < y)
\end{align*}

\end{frame}

\bibliographystyle{plain}
\bibliography{Database}

\end{document}
